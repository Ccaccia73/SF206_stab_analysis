%%%%%%%%%%%%%%%%%%%%%%%%%%%%%%%%%%%%%%%%%
% Arsclassica Article
% LaTeX Template
% Version 1.1 (10/6/14)
%
% This template has been downloaded from:
% http://www.LaTeXTemplates.com
%
% Original author:
% Lorenzo Pantieri (http://www.lorenzopantieri.net) with extensive modifications by:
% Vel (vel@latextemplates.com)
%
% License:
% CC BY-NC-SA 3.0 (http://creativecommons.org/licenses/by-nc-sa/3.0/)
%
%%%%%%%%%%%%%%%%%%%%%%%%%%%%%%%%%%%%%%%%%

%----------------------------------------------------------------------------------------
%	PACKAGES AND OTHER DOCUMENT CONFIGURATIONS
%----------------------------------------------------------------------------------------

\documentclass[
10pt, % Main document font size
a4paper, % Paper type, use 'letterpaper' for US Letter paper
oneside, % One page layout (no page indentation)
%twoside, % Two page layout (page indentation for binding and different headers)
headinclude,footinclude, % Extra spacing for the header and footer
BCOR5mm, % Binding correction
]{scrartcl}

\input{structure.tex} % Include the structure.tex file which specified the document structure and layout

\usepackage[italian]{babel}

\hyphenation{Fortran hy-phen-ation} % Specify custom hyphenation points in words with dashes where you would like hyphenation to occur, or alternatively, don't put any dashes in a word to stop hyphenation altogether

%----------------------------------------------------------------------------------------
%	TITLE AND AUTHOR(S)
%----------------------------------------------------------------------------------------

\title{\normalfont\spacedallcaps{Analisi Stabilizzatore SIAI MARCHETTI SF-260}} % The article title

\author{\spacedlowsmallcaps{Claudio Caccia}} % The article author(s) - author affiliations need to be specified in the AUTHOR AFFILIATIONS block

\date{\today} % An optional date to appear under the author(s)

%----------------------------------------------------------------------------------------

\begin{document}

%----------------------------------------------------------------------------------------
%	HEADERS
%----------------------------------------------------------------------------------------

\renewcommand{\sectionmark}[1]{\markright{\spacedlowsmallcaps{#1}}} % The header for all pages (oneside) or for even pages (twoside)
%\renewcommand{\subsectionmark}[1]{\markright{\thesubsection~#1}} % Uncomment when using the twoside option - this modifies the header on odd pages
\lehead{\mbox{\llap{\small\thepage\kern1em\color{halfgray} \vline}\color{halfgray}\hspace{0.5em}\rightmark\hfil}} % The header style

\pagestyle{scrheadings} % Enable the headers specified in this block

%----------------------------------------------------------------------------------------
%	TABLE OF CONTENTS & LISTS OF FIGURES AND TABLES
%----------------------------------------------------------------------------------------

\maketitle % Print the title/author/date block

\setcounter{tocdepth}{2} % Set the depth of the table of contents to show sections and subsections only

\tableofcontents % Print the table of contents

\listoffigures % Print the list of figures

\listoftables % Print the list of tables

%----------------------------------------------------------------------------------------
%	ABSTRACT
%----------------------------------------------------------------------------------------

\section*{Oggetto} % This section will not appear in the table of contents due to the star (\section*)

Scopo dell'analisi consiste nell'effettuare una modellazione ad elementi finiti dello stabilizzatore del velivolo \emph{SIAI Marchetti SF-260} (cfr. Figura \ref{fig:stab260}) nelle condizioni di carico specificate nella documentazione fornita, arrivando a determinare la configurazione deformata (freccia massima, angolo di torsione, etc.) ed una stima dello stato di sforzo nella struttura.

\begin{figure}[tb]
	\centering 
	\includegraphics[width=0.9\columnwidth]{stab.jpg} 
	\caption[foto stabilizzatore]{Foto dello stabilizzatore} % The text in the square bracket is the caption for the list of figures while the text in the curly brackets is the figure caption
	\label{fig:stab260} 
\end{figure}

%\url{http://www.mcescher.com/}
%----------------------------------------------------------------------------------------
%	AUTHOR AFFILIATIONS
%----------------------------------------------------------------------------------------

%{\let\thefootnote\relax\footnotetext{* \textit{Department of Biology, University of Examples, London, United Kingdom}}}

%{\let\thefootnote\relax\footnotetext{\textsuperscript{1} \textit{Department of Chemistry, University of Examples, London, United Kingdom}}}

%----------------------------------------------------------------------------------------

\newpage % Start the article content on the second page, remove this if you have a longer abstract that goes onto the second page

%----------------------------------------------------------------------------------------
%	INTRODUCTION
%----------------------------------------------------------------------------------------

\section{Strumenti}

%A statement\footnote{Example of a footnote} requiring citation \cite{Figueredo:2009dg}.
%Some mathematics in the text: $\cos\pi=-1$ and $\alpha$.

L'analisi \emph{FEM} \`{e} stata eseguita utilizzando i seguenti sotfware di modellazione e calcolo:

\begin{itemize}
	\item \textbf{Code-Aster} \url{http://code-aster.org} (vers. 13.1) per la modellazione ad elementi finiti,
	\item \textbf{{Salome}} \url{http://www.salome-platform.org} (vers. 7.7.1) per modellazione geometrica e mesh
	\item \textbf{ParaView} \url{http://www.paraview.org}(vers. 4.7) per il post-processing
	\item \textbf{Python} per ogni attivit\`{a} di calcolo
\end{itemize}

Il codice scritto per la realizzazione delle analisi \`{e} disponibile al seguente indirizzo: \url{https://github.com/Ccaccia73/SF206_stab_analysis}



\section{Panoramica delle analisi}

Lo studio comprende:

\begin{itemize}
	\item Unit\`{a} di misura: \textbf{SI} [N, mm, t]
	\item Numero di geometrie: \textbf{1}
	\item Numero di mesh: \textbf{4}
	\item Numero di analisi FEM: \textbf{7}
\end{itemize}

Ad esse si aggiungono alcune analisi complementari per la verifica e l'ottimizzazione degli elementi ed ulteriori calcoli per la verifica della validit\`{a} delle analisi compiute.


\section{Geometria}

Le parti della geometria sono rappresentate in Figura \ref{fig:geom}, da cui sono stati soppressi i pannelli superiori per rendere visibili gli elementi:

\begin{itemize}
	\item Longherone principale (blu)
	\item Longherone anteriore (verde)
	\item Centine (grigio)
	\item Pannelli posteriori (rosso)
	\item Pannelli anteriori (giallo)
\end{itemize}

Il sistema di riferimento \`{e} illustrato in figura: l'origine \`{e} in mezzeria a met\`{a} longherone, lo stabilizzatore \`{e} diretto lungo \emph{z}, l'asse \emph{x} \`{e} diretto verso la coda e \emph{y} verso l'alto.

\begin{figure}[tb]
	\centering 
	\includegraphics[width=0.9\columnwidth]{geom.jpg} 
	\caption[Geometria stabilizzatore]{Geometria stab. SF260} % The text in the square bracket is the caption for the list of figures while the text in the curly brackets is the figure caption
	\label{fig:geom} 
\end{figure}

\section{Mesh}

\newpage

\begin{enumerate}[noitemsep] % [noitemsep] removes whitespace between the items for a compact look
\item First item in a list
\item Second item in a list
\item Third item in a list
\end{enumerate}

%------------------------------------------------

\subsection{Paragraphs}

\lipsum[6] % Dummy text

\paragraph{Paragraph Description} \lipsum[7] % Dummy text

\paragraph{Different Paragraph Description} \lipsum[8] % Dummy text

%------------------------------------------------

\subsection{Math}

\lipsum[4] % Dummy text

\begin{equation}
\cos^3 \theta =\frac{1}{4}\cos\theta+\frac{3}{4}\cos 3\theta
\label{eq:refname2}
\end{equation}

\lipsum[5] % Dummy text

\begin{definition}[Gauss] 
To a mathematician it is obvious that
$\int_{-\infty}^{+\infty}
e^{-x^2}\,dx=\sqrt{\pi}$. 
\end{definition} 

\begin{theorem}[Pythagoras]
The square of the hypotenuse (the side opposite the right angle) is equal to the sum of the squares of the other two sides.
\end{theorem}

\begin{proof} 
We have that $\log(1)^2 = 2\log(1)$.
But we also have that $\log(-1)^2=\log(1)=0$.
Then $2\log(-1)=0$, from which the proof.
\end{proof}

%----------------------------------------------------------------------------------------
%	RESULTS AND DISCUSSION
%----------------------------------------------------------------------------------------

\section{Results and Discussion}

\lipsum[10] % Dummy text

%------------------------------------------------

\subsection{Subsection}

\lipsum[11] % Dummy text

\subsubsection{Subsubsection}

\lipsum[12] % Dummy text

\begin{description}
\item[Word] Definition
\item[Concept] Explanation
\item[Idea] Text
\end{description}

\lipsum[12] % Dummy text

\begin{itemize}[noitemsep] % [noitemsep] removes whitespace between the items for a compact look
\item First item in a list
\item Second item in a list
\item Third item in a list
\end{itemize}

\subsubsection{Table}

\lipsum[13] % Dummy text

\begin{table}[hbt]
\caption{Table of Grades}
\centering
\begin{tabular}{llr}
\toprule
\multicolumn{2}{c}{Name} \\
\cmidrule(r){1-2}
First name & Last Name & Grade \\
\midrule
John & Doe & $7.5$ \\
Richard & Miles & $2$ \\
\bottomrule
\end{tabular}
\label{tab:label}
\end{table}

Reference to Table~\vref{tab:label}. % The \vref command specifies the location of the reference

%------------------------------------------------

\subsection{Figure Composed of Subfigures}

Reference the figure composed of multiple subfigures as Figure~\vref{fig:esempio}. Reference one of the subfigures as Figure~\vref{fig:ipsum}. % The \vref command specifies the location of the reference

\lipsum[15-18] % Dummy text

\begin{figure}[tb]
\centering
\subfloat[A city market.]{\includegraphics[width=.45\columnwidth]{Lorem}} \quad
\subfloat[Forest landscape.]{\includegraphics[width=.45\columnwidth]{Ipsum}\label{fig:ipsum}} \\
\subfloat[Mountain landscape.]{\includegraphics[width=.45\columnwidth]{Dolor}} \quad
\subfloat[A tile decoration.]{\includegraphics[width=.45\columnwidth]{Sit}}
\caption[A number of pictures.]{A number of pictures with no common theme.} % The text in the square bracket is the caption for the list of figures while the text in the curly brackets is the figure caption
\label{fig:esempio}
\end{figure}

%----------------------------------------------------------------------------------------
%	BIBLIOGRAPHY
%----------------------------------------------------------------------------------------

\renewcommand{\refname}{\spacedlowsmallcaps{References}} % For modifying the bibliography heading

\bibliographystyle{unsrt}

\bibliography{sample.bib} % The file containing the bibliography

%----------------------------------------------------------------------------------------

\end{document}