%%%%%%%%%%%%%%%%%%%%%%%%%%%%%%%%%%%%%%%%%
% Arsclassica Article
% LaTeX Template
% Version 1.1 (10/6/14)
%
% This template has been downloaded from:
% http://www.LaTeXTemplates.com
%
% Original author:
% Lorenzo Pantieri (http://www.lorenzopantieri.net) with extensive modifications by:
% Vel (vel@latextemplates.com)
%
% License:
% CC BY-NC-SA 3.0 (http://creativecommons.org/licenses/by-nc-sa/3.0/)
%
%%%%%%%%%%%%%%%%%%%%%%%%%%%%%%%%%%%%%%%%%

%----------------------------------------------------------------------------------------
%	PACKAGES AND OTHER DOCUMENT CONFIGURATIONS
%----------------------------------------------------------------------------------------

\documentclass[
10pt, % Main document font size
a4paper, % Paper type, use 'letterpaper' for US Letter paper
oneside, % One page layout (no page indentation)
%twoside, % Two page layout (page indentation for binding and different headers)
headinclude,footinclude, % Extra spacing for the header and footer
BCOR5mm, % Binding correction
]{scrartcl}

\input{structure.tex} % Include the structure.tex file which specified the document structure and layout

\usepackage[italian]{babel}

\hyphenation{Fortran hy-phen-ation} % Specify custom hyphenation points in words with dashes where you would like hyphenation to occur, or alternatively, don't put any dashes in a word to stop hyphenation altogether

%----------------------------------------------------------------------------------------
%	TITLE AND AUTHOR(S)
%----------------------------------------------------------------------------------------

\title{\normalfont\spacedallcaps{Analisi Stabilizzatore SIAI MARCHETTI SF-260}} % The article title

\author{\spacedlowsmallcaps{Claudio Caccia}} % The article author(s) - author affiliations need to be specified in the AUTHOR AFFILIATIONS block

\date{\today} % An optional date to appear under the author(s)

%----------------------------------------------------------------------------------------

\begin{document}

%----------------------------------------------------------------------------------------
%	HEADERS
%----------------------------------------------------------------------------------------

\renewcommand{\sectionmark}[1]{\markright{\spacedlowsmallcaps{#1}}} % The header for all pages (oneside) or for even pages (twoside)
%\renewcommand{\subsectionmark}[1]{\markright{\thesubsection~#1}} % Uncomment when using the twoside option - this modifies the header on odd pages
\lehead{\mbox{\llap{\small\thepage\kern1em\color{halfgray} \vline}\color{halfgray}\hspace{0.5em}\rightmark\hfil}} % The header style

\pagestyle{scrheadings} % Enable the headers specified in this block

%----------------------------------------------------------------------------------------
%	TABLE OF CONTENTS & LISTS OF FIGURES AND TABLES
%----------------------------------------------------------------------------------------

\maketitle % Print the title/author/date block

\setcounter{tocdepth}{2} % Set the depth of the table of contents to show sections and subsections only

\tableofcontents % Print the table of contents

\listoffigures % Print the list of figures

\listoftables % Print the list of tables

%----------------------------------------------------------------------------------------
%	ABSTRACT
%----------------------------------------------------------------------------------------

\section*{Oggetto} % This section will not appear in the table of contents due to the star (\section*)

Scopo dell'analisi consiste nell'effettuare una modellazione ad elementi finiti dello stabilizzatore del velivolo \emph{SIAI Marchetti SF-260} (cfr. Figura \ref{fig:stab260}) nelle condizioni di carico specificate nella documentazione fornita, arrivando a determinare la configurazione deformata (freccia massima, angolo di torsione, etc.) ed una stima dello stato di sforzo nella struttura.

\begin{figure}[tb]
	\centering 
	\includegraphics[width=0.9\columnwidth]{stab.jpg} 
	\caption[foto stabilizzatore]{Foto dello stabilizzatore} % The text in the square bracket is the caption for the list of figures while the text in the curly brackets is the figure caption
	\label{fig:stab260} 
\end{figure}

%\url{http://www.mcescher.com/}
%----------------------------------------------------------------------------------------
%	AUTHOR AFFILIATIONS
%----------------------------------------------------------------------------------------

%{\let\thefootnote\relax\footnotetext{* \textit{Department of Biology, University of Examples, London, United Kingdom}}}

%{\let\thefootnote\relax\footnotetext{\textsuperscript{1} \textit{Department of Chemistry, University of Examples, London, United Kingdom}}}

%----------------------------------------------------------------------------------------

\newpage % Start the article content on the second page, remove this if you have a longer abstract that goes onto the second page

%----------------------------------------------------------------------------------------
%	INTRODUCTION
%----------------------------------------------------------------------------------------

\section{Strumenti}

%A statement\footnote{Example of a footnote} requiring citation \cite{Figueredo:2009dg}.
%Some mathematics in the text: $\cos\pi=-1$ and $\alpha$.

L'analisi \emph{FEM} \`{e} stata eseguita utilizzando i seguenti sotfware di modellazione e calcolo:

\begin{itemize}
	\item \textbf{Code-Aster} \url{http://code-aster.org} (vers. 13.1) per la modellazione ad elementi finiti,
	\item \textbf{{Salome}} \url{http://www.salome-platform.org} (vers. 7.7.1) per modellazione geometrica e mesh
	\item \textbf{ParaView} \url{http://www.paraview.org}(vers. 4.7) per il post-processing
	\item \textbf{Python} per ogni attivit\`{a} di calcolo
\end{itemize}

Il codice scritto per la realizzazione delle analisi \`{e} disponibile al seguente indirizzo: \url{https://github.com/Ccaccia73/SF206_stab_analysis}



\section{Panoramica delle analisi}

Unit\`{a} di misura adottate: \textbf{SI} [N, mm, t]\\
\\

Lo studio comprende:

\begin{itemize}
	\item Numero di geometrie: \textbf{1}
	\item Numero di mesh: \textbf{3}
	\item Numero di analisi FEM: \textbf{6}
\end{itemize}

Ad esse si aggiungono alcune analisi complementari per la verifica e l'ottimizzazione dei componenti ed ulteriori calcoli per la verifica della validit\`{a} delle analisi compiute.


\section{Geometria}

Le parti della geometria sono rappresentate in Figura \ref{fig:geom}, da cui sono stati soppressi i pannelli superiori per rendere visibili gli elementi:

\begin{itemize}
	\item Longherone principale (blu)
	\item Longherone anteriore (verde)
	\item Centine (grigio)
	\item Pannelli posteriori (rosso)
	\item Pannelli anteriori (giallo)
\end{itemize}

Il sistema di riferimento è illustrato in figura: l'origine è in mezzeria a metà longherone, lo stabilizzatore è diretto lungo \emph{z}, l'asse \emph{x} è diretto verso la coda e \emph{y} verso l'alto.

\begin{figure}[tb]
	\centering 
	\includegraphics[width=0.9\columnwidth]{geom.jpg} 
	\caption[Geometria stabilizzatore]{Geometria stab. SF260} % The text in the square bracket is the caption for the list of figures while the text in the curly brackets is the figure caption
	\label{fig:geom} 
\end{figure}


\section{Caratterizzazione degli elementi}

Prima di procedere all'analisi è necessario caratterizzare opportunamente le componenti dello stabilizzatore, con l'intento di semplificare la struttura il più possibile senza perdere significatività dei risultati. 

\subsection{Centine}

La struttura delle 5 centine è composta da un'anima spessa 0.64mm e due flange larghe 15mm. La numerazione parte dalla mezzeria. Le prime 4 centine sono caratterizzate da fori di alleggerimento.\\
Per semplificare il modello si è scelto di modellare la centina con le seguenti caratteristiche:

\begin{itemize}
	\item Sono stati trascurati i fori di alleggerimento,
	\item l'anima è stata modellata con elementi di tipo \emph{shell},
	\item le flange sono state modellate con elementi di tipo \emph{beam}.
\end{itemize} 

In questo modo si rischia di perdere informazioni relative ad eventuali concentrazioni di sforzo in corrispondenza dei fori, inoltre la rigidezza flesso-torsionale della centina viene leggermente modificata.\\
Per ovviare a questi inconvenienti sono state condotte analisi preliminari sulle centine al fine di:

\begin{itemize}
	\item definire uno spessore equivalente che replichi la stessa rigidezza flessionale della centina (ritenuta l'aspetto più significativo),
	\item analizzare l'andamento degli sforzi nella centina in condizioni di carico semplici per avere un'indicazione su eventuali concentrazioni di sforzi.
\end{itemize}

Per ogni centina sono state condotte le seguenti analisi:

\begin{itemize}
	\item centina originale incastrata in corrispondenza del longherone principale, movimento lungo \emph{z} impedito e carico unitario in estremità lungo \emph{y} (cedevolezza flessionale),
	\item centina originale incastrata in corrispondenza del longherone principale e coppia unitaria in estremità diretta secondo \emph{x} (cedevolezza torsionale),
	\item centina semplificata caricata nello stesso modo per determinare lo spessore equivalente che garantisca la stessa cedevolezza flessionale, infine verifica della cedevolezza torsionale. 
\end{itemize}

Nelle figure \ref{fig:C1_comp} e \ref{fig:C1plot}  si riportano ad esempio i risultati ottenuti per la \emph{centina 1}.

\begin{figure}[tb]
	\centering
	\subfloat[Centina 1: deformazione flessionale]{\includegraphics[width=.90\columnwidth]{C1_comp_DY.png}\label{fig:C1_DY}} \\
	\subfloat[Centina 1: VM stress]{\includegraphics[width=.90\columnwidth]{C1_comp_VMIS.png}\label{fig:C1_VM}} \\
	\caption[Centina 1 completa]{Analisi Centina 1 completa} % The text in the square bracket is the caption for the list of figures while the text in the curly brackets is the figure caption
	\label{fig:C1_comp}
\end{figure}

\begin{figure}[tb]
	\centering
	\subfloat[Centina 1: VM stress root]{\includegraphics[width=.90\columnwidth]{C1_VM_root}\label{fig:C1_root}} \\
	\subfloat[Centina 1: VM stress @53.5mm]{\includegraphics[width=.90\columnwidth]{C1_VM_535}\label{fig:C1_535}} \\
	\caption[Centina 1 confronto VM stress]{Centina 1 confronto VM stress} % The text in the square bracket is the caption for the list of figures while the text in the curly brackets is the figure caption
	\label{fig:C1plot}
\end{figure}

La figura \ref{fig:C1plot} fornisce un'indicazione sull'incremento degli sforzi nella centina (all'incastro ed in corrispondenza del primo foro, a $53.5mm$ dall'incastro) per effetto della presenza dei fori.


\begin{table}[bt]
	\caption{Risultati delle analisi sulle centina}
	\centering
	\begin{tabular}{rcccccc}
		\toprule
		& \multicolumn{3}{c}{Completo} & \multicolumn{3}{c}{Semplificato}\\
		\cmidrule(r){2-4} \cmidrule(l){5-7} 
		& DY  & DRX & massa & spessore & DRX  & massa \\
		& $[\mu m/N]$ & $[\mu rad/Nmm]$ & $[g]$ & $[\mu m]$ & $[\mu rad/Nmm]$ & $[g]$ \\
		\midrule
		C1 & $16.18$ & $771$ & $105$ & $513$ & $458$ & $95$ \\
		C2 & $16.52$ & $782$ & $84$ & $497$ & $488$ & $75$ \\
		C3 & $14.10$ & $880$ & $67$ & $550$ & $372$ & $60$ \\
		C4 & $13.68$ & $841$ & $45$ & $526$ & $386$ & $39$ \\
		C5 & $13.26$ & $849$ & $29$ & $508$ & $356$ & $23$ \\
		\bottomrule
	\end{tabular}
	\label{tab:centine}
\end{table}

La tabella \ref{tab:centine} mostra come, a parità di rigidezza flessionale, le centine semplificate risultino più rigide torsionalmente e di poco più leggere. Si ritiene la semplificazione comunque accettabile e si utilizzano tali valori nell'analisi.

\newpage

\subsection{Longherone anteriore}

Il longherone anteriore (LA) è lungo $740mm$, ha uno spessore di $1.6mm$ ed è disposto dalla mezzeria fino alla centina 3. Come le centine ha una struttura a \textquotedblleft C\textquotedblright con fori di alleggerimento. Presenta inoltre un irrigidimento nella parte centrale.
Si operano semplificazioni simili a quelle adottate per le centine:

\begin{itemize}
	\item Modellizzazione dell'anima a \emph{shell} e delle flange a \emph{beam}
	\item eliminazione dei fori e determinazione di uno spessore equivalente
\end{itemize}

Nelle figure \ref{fig:FS_compl} e \ref{fig:FS_simp} si riportano i risultati ottenuti.


\begin{figure}[tb]
	\centering
	\subfloat[LA: sforzi ed azioni]{\includegraphics[width=.90\columnwidth]{FS_compl_VM.png}\label{fig:FSc_VM}} \\
	\subfloat[LA: deformazione]{\includegraphics[width=.90\columnwidth]{FS_compl_DY.png}\label{fig:FSc_DY}} \\
	\caption[Longherone anteriore completo]{Analisi longherone anteriore completo} % The text in the square bracket is the caption for the list of figures while the text in the curly brackets is the figure caption
	\label{fig:FS_compl}
\end{figure}


\begin{figure}[tb]
	\centering
	\subfloat[LA: sforzi ed azioni]{\includegraphics[width=.90\columnwidth]{FS_simp_str.png}\label{fig:FSs_VM}} \\
	\subfloat[LA: deformazione]{\includegraphics[width=.90\columnwidth]{FS_simp_DY.png}\label{fig:FSs_DY}} \\
	\caption[Longherone anteriore semplificato]{Analisi longherone anteriore semplificato} % The text in the square bracket is the caption for the list of figures while the text in the curly brackets is the figure caption
	\label{fig:FS_simp}
\end{figure}

Il calcolo teorico di una mensola incastrata linearmente rastremata con le stesse dimensioni del longherone anteriore fornisce una cedevolezza flessionale di $8\cdot10^{-3} mm/N$ contro i $7.89\cdot10^{-3} mm/N$ del modello, confermando la validità della simulazione.\\
La tabella \ref{tab:FS} riporta la sintesi dei risultati ottenuti.

\begin{table}[hbt]
	\caption{Risultati delle analisi sul longherone anteriore}
	\centering
	\begin{tabular}{rcccccc}
		\toprule
		& \multicolumn{3}{c}{Completo} & \multicolumn{3}{c}{Semplificato}\\
		\cmidrule(r){2-4} \cmidrule(l){5-7} 
		& DY  & DRZ & massa & spessore & DRZ  & massa \\
		& $[\mu m/N]$ & $[\mu rad/Nmm]$ & $[g]$ & $[mm]$ & $[\mu rad/Nmm]$ & $[g]$ \\
		\midrule
		LA & $7.89$ & $114$ & $454$ & $1.471$ & $128$ & $429$ \\
		\bottomrule
	\end{tabular}
	\label{tab:FS}
\end{table}


\subsection{Longherone principale}

Il longherone principale (LP) è privo di fori di alleggerimento, ha degli irrigidimenti in corrispondenza della zona centrale e tra le centine 4 e 5, inoltre ha un corrente a \textquotedblleft U\textquotedblright \ per tutta la lunghezza ed un corrente a \textquotedblleft L\textquotedblright fino alla distanza di $510mm$. \`{E} stato modellato direttamente nel sistema completo, con le seguenti caratteristiche:

\begin{itemize}
	\item spessori (elementi di tipo \emph{shell}) nelle 5 sezioni del LP:
	\begin{itemize}
		\item $1.6+2mm$ nella sezione 1, tenendo condo in modo semplificato dell'irrigidimento
		\item $1.6mm$ nelle sezioni 2,3 e 4
		\item $1.6+2.9mm$ nella sezione 5, tenendo conto dell'irrigidimento
	\end{itemize}
	\item elementi di tipo \emph{beam}:
	\begin{itemize}
		\item corrente a \textquotedblleft U\textquotedblright \ più corrente a \textquotedblleft L\textquotedblright nelle sezioni 1 e 2, tenendo conto in modo semplificato del corrente a \textquotedblleft L\textquotedblright, più lungo e rastremato
		\item corrente a \textquotedblleft U\textquotedblright \ nelle sezioni 3 e 4
		\item corrente a \textquotedblleft U\textquotedblright \ più flangia da $2.9 \times 25 mm$ nella sezione 5
	\end{itemize}
\end{itemize}

\newpage

\subsection{Correnti}

Oltre ai correnti nel longherone principale sono presenti dei correnti a \textquotedblleft L\textquotedblright \ nelle sezioni 4 e 5, oltre il longherone principale. Vengono modellati con elementi di tipo \emph{beam}.

\subsection{Pannelli}

Tutti i pannelli vengono modellati con elementi di tipo \emph{shell}, spessi $0.64mm$ nella parte anteriore (fino al LA o ai correnti) e $0.81mm$ nella parte posteriore. 


\newpage

\section{Mesh}

La mesh è stata generata per ogni componente, realizzando poi un'unica \emph{compound mesh} e verificando che gli elementi (nodi e spigoli) fossero coincidenti.
La mesh è composta da 7271 elementi lineari di cui:

\begin{itemize}
	\item $7138$ elementi di tipo \emph{QUAD4}
	\item $133$ elementi di tipo \emph{TRIA3}
\end{itemize} 

\begin{figure}[tb]
	\centering
	\subfloat[mesh: pannelli]{\includegraphics[width=.98\columnwidth]{mesh1.jpg}\label{fig:mesh1}} \\
	\subfloat[mesh: centine e longheroni]{\includegraphics[width=.98\columnwidth]{mesh2.jpg}\label{fig:mesh2}} \\
	\subfloat[mesh: elementi beam]{\includegraphics[width=.98\columnwidth]{mesh3.jpg}\label{fig:mesh3}} \\
	\caption[Mesh Stabilizzatore SF260]{mesh dello stabilizzatore} % The text in the square bracket is the caption for the list of figures while the text in the curly brackets is the figure caption
	\label{fig:mesh}
\end{figure}


\newpage


\section{Analisi Preliminari}

Prima di procedere alle analisi con i carichi dovuti a raffica e manovra, si è deciso di eseguire dei test preliminari con carichi semplificati, in modo da poter eseguire un confronto con un modello analitico a semiguscio e verificare così la validità del modello finora realizzato. Per quanto riguarda i vincoli sono state adottate le condizioni descritte in \ref{sec:vincoli}.\\
Sono stati eseguiti due test:

\begin{itemize}
	\item Momento torcente $M_z = 1000 \ Nmm$ applicato alla centina 5 tramite un elemento di tipo \emph{RBE3}	
	\item Momento flettente $M_x = 1000 \ Nmm$ applicato alla centina 5 tramite un elemento di tipo \emph{RBE3}
\end{itemize}

Sono stati poi misurati gli spostamenti medi e le rotazioni medie della centina 5 e confrontati con i risultati analitici.

\subsection{Simulazione momento torcente}

La simulazione fornisce un valore medio di rotazione intorno a \emph{z} pari a $3.104\cdot 10^{-5} rad$.

La figura \ref{fig:Mztest} mostra l'andamento della deformazione.

\begin{figure}[tb]
	\centering 
	\includegraphics[width=0.9\columnwidth]{Mz_test.png} 
	\caption[Test Mz]{Test momento torcente} % The text in the square bracket is the caption for the list of figures while the text in the curly brackets is the figure caption
	\label{fig:Mztest} 
\end{figure}



\subsection{Simulazione momento flettente}

La simulazione fornisce un valore medio di rotazione intorno a \emph{x} pari a $1.548\cdot 10^{-5} rad$ ed uno spostamento medio lungo \emph{y} di $-1.01 \cdot 10^{-2}mm$.

La figura \ref{fig:Mxtest} mostra l'andamento della deformazione.

\begin{figure}[htb]
	\centering 
	\includegraphics[width=0.9\columnwidth]{Mx_test.png} 
	\caption[Test Mx]{Test momento flettente} % The text in the square bracket is the caption for the list of figures while the text in the curly brackets is the figure caption
	\label{fig:Mxtest} 
\end{figure}

\newpage

\section{Calcolo tramite teoria a semiguscio}

Per il corso di \emph{Strutture Aerospaziali} avevo sviluppato un programma per il calcolo dei parametri delle sezioni a semiguscio. Il codice è disponibile qui: \url{https://github.com/Ccaccia73/semimonocoque}. In particolare, i calcoli eseguiti per lo stabilizzatore sono riportati qui: \url{http://nbviewer.jupyter.org/github/ccaccia73/semimonocoque/blob/master/10_Stab_SF260.ipynb}

\subsection{Modello a semiguscio}

Lo stabilizzatore è stato suddiviso in 5 sezioni. Come dimensione in direzione \emph{x} è stato considerato il valore medio del tratto in esame. Si è trascurato l'angolo di inclinazione della centina 1 e sono state considerate le aree e gli spessori determinati in precedenza per il modello FEM. Per i longheroni si è utilizzato il metodo dell'\emph{area collaborante}.

La figura \ref{fig:semiguscio} illustra le 5 sezioni descritte nel modello a semiguscio. I valori dei parametri sono visibili al link sopra.


\begin{figure}[tb]
	\centering
	\subfloat[Sezione 1]{\includegraphics[width=.485\columnwidth]{sect1.jpg}} \quad
	\subfloat[Sezione 2]{\includegraphics[width=.485\columnwidth]{sect2.jpg}} \\
	\subfloat[Sezione 3]{\includegraphics[width=.485\columnwidth]{sect3.jpg}} \quad
	\subfloat[Sezione 4]{\includegraphics[width=.485\columnwidth]{sect4.jpg}} \\
	\subfloat[Sezione 5]{\includegraphics[width=.485\columnwidth]{sect5.jpg}} \\
	\caption[modello a semiguscio]{Sezioni a semiguscio dello stabilizzatore} % The text in the square bracket is the caption for the list of figures while the text in the curly brackets is the figure caption
	\label{fig:semiguscio}
\end{figure}


\subsection{Calcolo sezione sottoposta momento torcente}

Il modello a semiguscio fornisce una stima della rigidezza torsionale $J_{ti}$ per ogni tratto dello stabilizzatore. L'angolo totale di torsione viene calcolato tramite:

\begin{equation}
	\phi = \frac{M_z}{G} \cdot \sum_{i=1}^{5} \frac{L_i}{J_{ti}}
	\label{eq:sg_torque}
\end{equation}

con $L_i$ lunghezza del tratto i-esimo.\\
Il calcolo fornisce un valore di $\phi = 3.021 \cdot 10^{-5}$ rad.



\subsection{Calcolo sezione sottoposta momento flettente}

Il modello a semiguscio fornisce una stima della rigidezza flessionale $J_{xi}$ per ogni tratto dello stabilizzatore. L'angolo di flessione all'estremità viene calcolato tramite:


\begin{equation}
\theta = \frac{M_x}{E} \cdot \sum_{i=1}^{5} \frac{L_i}{J_{xi}}
\label{eq:sg_tip_flex}
\end{equation}

Il calcolo fornisce un valore di $\theta = 1.688 \cdot 10^{-5}$ rad.

Lo spostamento verticale $\delta$ dell'estremità deriva dalle seguenti espressioni:

\begin{equation}
\delta = \int_{0}^{l}M_x \cdot \frac{z-l}{EJ_x(z)}dz
\label{eq:sg_tip1}
\end{equation}

Suddividendo l'integrale nei tratti a $J_x$ costante si ottiene:

\begin{equation}
\delta = \sum_{i=1}^{5} \frac{l_i}{EJ_{xi}} \left\{ \frac{\bar{z}_i}{2}  - l  \right\} 
\label{eq:sg_tip2}
\end{equation}

Con $l_i$ lunghezza del tratto i-esimo e $\bar{z}_i$ posizione di metà tratto.\\
Il calcolo fornisce un valore di $\delta = -0.0085mm$.


\subsection{Confronto dei risultati:}

\begin{table}[hbt]
	\caption{Confronto risultati}
	\centering
	\begin{tabular}{lcc}
		\toprule
		& \multicolumn{2}{c}{Valori} \\
		\cmidrule(r){2-3}
		 & SG & FEM \\
		\midrule
		tors. $\phi \ [\mu rad]$ & $30.21$ & $31.04$ \\
		flex. $\theta \ [\mu rad]$ & $16.88$ & $15.48$ \\
		flex. $\delta \ [\mu m]$ & $-85$ & $-101$ \\
		\bottomrule
	\end{tabular}
	\label{tab:comp}
\end{table}

La rispondenza tra il modello ad elementi finiti ed il modello a semiguscio viene ritenuta buona, pertanto si considera il modello \emph{FEM} affidabile.


\section{Vincoli}
\label{sec:vincoli}

Essendo la struttura simmetrica, caricata in modo simmetrico, è possibile analizzare solo la semiapertura dello stabilizzatore. Si è pertanto scelto di vincolare tramite incastro i due longheroni. In questo modo si mantiene una struttura semplice, si perdono però eventuali informazioni relative alla reale condizione di vincolo tra fusoliera e stabilizzatore, che può essere oggetto di un'analisi specifica.


\section{Carichi}

Vengono considerate due condizioni di carico ritenute significative, \emph{manovra} e \emph{raffica}. Di queste sono disponibili le azioni interne in diversi punti dello stabilizzatore (cfr. figura \ref{fig:az_dati}). Vengono qui riportati i risultati ottenuti dall'analisi dei carichi, i calcoli effettuati sono riportati qui:\\ \url{http://nbviewer.jupyter.org/github/ccaccia73/SF206_stab_analysis/blob/master/04_carichi/actions.ipynb}.

\begin{figure}[htb]
	\centering
	\subfloat[manovra]{\includegraphics[width=.485\columnwidth]{01_dati_manovra.jpg}} \
	\subfloat[raffica]{\includegraphics[width=.485\columnwidth]{01_dati_raffica.jpg}} \\
	\caption[azioni interne: dati]{Azioni interne: dati} % The text in the square bracket is the caption for the list of figures while the text in the curly brackets is the figure caption
	\label{fig:az_dati}
\end{figure}

\`{E} presente un carico concentrato a $1200 \ mm$ dalla mezzeria, in corrispondenza dell'attacco dell'equilibratore. Tale carico corrisponde ad una forza concentrata pari a:

\begin{itemize}
	\item $-841.4 \ N$ in caso di manovra
	\item $-172.8 \ N $ in caso di raffica
\end{itemize}

Tale forza ha un braccio di $40mm$ dal longherone principale e viene modellata tramite un elemento \emph{RBE3} che agisce su una regione del longherone principale corrispondente alla piastra di connessione con l'equilibratore.\\

Le componenti rimanenti sono considerate forze distribuite e si può verificare la relazione $T_y = \frac{dM_x}{dz}$ (cfr. figura \ref{fig:az_distr}, a meno del segno):

\begin{figure}[tb]
	\centering
	\subfloat[az. distribuite manovra]{\includegraphics[width=.485\columnwidth]{02_azioni_distribuite_manovra.jpg}} \
	\subfloat[az. distribuite raffica]{\includegraphics[width=.485\columnwidth]{02_azioni_distribuite_raffica.jpg}} \\
	\subfloat[verifica manovra]{\includegraphics[width=.485\columnwidth]{03_verifica_manovra.jpg}} \
	\subfloat[verifica raffica]{\includegraphics[width=.485\columnwidth]{03_verifica_raffica.jpg}} \\
	\caption[azioni interne distribuite]{Azioni interne distribuite} % The text in the square bracket is the caption for the list of figures while the text in the curly brackets is the figure caption
	\label{fig:az_distr}
\end{figure}

Per caricare la struttura si possono adottare diverse filosofie. Si è scelto qui di considerare i seguenti casi:

\begin{enumerate}
	\item Caricare le \emph{centine}, un elemento \emph{RBE3} posizionato sul longherone principale per ogni centina ne carica il bordo, trasmettendo $T_y$ e $M_z$. Ricalca la teoria e fornisce un'approssimazione un po' grossolana di $T_y$ e $M_z$.
	\item Caricare con una forza superficiale costante ogni sezione dei \emph{longheroni}. Riproduce $T_y$ in modo fedele, tuttavia l'assenza del longherone anteriore nelle sezioni 4 e 5 rende molto approssimato $M_z$ nel tratto esterno dello stabilizzatore.
\end{enumerate} 

Di seguito si illustrano le due filosofie adottate.

\subsection{Carichi sulle centine}


La figura \ref{fig:az_centine} illustra l'andamento delle azioni interne ottenuto caricando le centine.


\begin{figure}[tb]
	\centering
	\subfloat[$T_y$ manovra]{\includegraphics[width=.485\columnwidth]{04_Ty_centine_manovra.jpg}} \
	\subfloat[$T_y$ raffica]{\includegraphics[width=.485\columnwidth]{04_Ty_centine_raffica.jpg}} \\
	\subfloat[$M_x$ manovra]{\includegraphics[width=.485\columnwidth]{05_Mx_centine_manovra.jpg}} \
	\subfloat[$M_x$ raffica]{\includegraphics[width=.485\columnwidth]{05_Mx_centine_raffica.jpg}} \\
	\subfloat[$M_z$ manovra]{\includegraphics[width=.485\columnwidth]{06_Mz_centine_manovra.jpg}} \
	\subfloat[$M_z$ raffica]{\includegraphics[width=.485\columnwidth]{06_Mz_centine_raffica.jpg}} \\
	\caption[Carichi sulle centine]{Carichi sulle centine} % The text in the square bracket is the caption for the list of figures while the text in the curly brackets is the figure caption
	\label{fig:az_centine}
\end{figure}


\subsection{Carichi sui longheroni}

La figura \ref{fig:az_longh} illustra l'andamento delle azioni interne ottenuto caricando i longheroni. Si nota che, scegliendo di caricare il solo longherone anteriore per dare momento torcente, l'azione $M_z$ è replicata in modo scadente. Tuttavia, essendo una zona a basso carico l'approssimazione potrebbe essere comunque adeguata.


\begin{figure}[tb]
	\centering
	\subfloat[$T_y$ manovra]{\includegraphics[width=.49\columnwidth]{07_Ty_distribuito_manovra.jpg}} \
	\subfloat[$T_y$ raffica]{\includegraphics[width=.49\columnwidth]{07_Ty_distribuito_raffica.jpg}} \\
	\subfloat[$M_x$ manovra]{\includegraphics[width=.49\columnwidth]{08_Mx_distrib_manovra.jpg}} \
	\subfloat[$M_x$ raffica]{\includegraphics[width=.49\columnwidth]{08_Mx_distrib_raffica.jpg}} \\
	\subfloat[$M_z$ manovra]{\includegraphics[width=.49\columnwidth]{09_Mz_distrib_manovra.jpg}} \
	\subfloat[$M_z$ raffica]{\includegraphics[width=.49\columnwidth]{09_Mz_distrib_raffica.jpg}} \\
	\caption[Carichi distribuiti sui longheorni]{Carichi distribuiti sui longheroni} % The text in the square bracket is the caption for the list of figures while the text in the curly brackets is the figure caption
	\label{fig:az_longh}
\end{figure}


\section{Materiale}

Tutta la struttura è realizzata in $A2024$, materiale elastico lineare ed isotropo, di cui sono state considerate le seguenti caratteristiche \ref{tab:mat}:

\begin{table}[hbt]
	\caption{Materiale}
	\centering
	\begin{tabular}{lcccc}
		\toprule
		& \multicolumn{4}{c}{Proprietà} \\
		\cmidrule(r){2-5}
		& E & $\nu$ & $\rho$ & $\sigma_y$  \\
		& $[MPa]$ &  & $[kg/m^3]$ & $[MPa]$  \\
		\midrule
		A2024 & $73100$ & $0.33$ & $2780$ & $324$ \\
		\bottomrule
	\end{tabular}
	\label{tab:mat}
\end{table}

Poiché risulta dal modello che la massa complessiva dello stabilizzatore è pari a $5.44 \ kg$, si decide di trascurare il peso proprio della struttura, così come il peso della vernice.

\newpage

\section{Analisi}

Sono state condotte 4 differenti analisi, una per ciascun tipo di carico e modalità di applicazione. La tabella \ref{tab:res} riporta la sintesi dei risultati ottenuti.

\begin{table}[bt]
	\caption{Sintesi dei risultati}
	\centering
	\begin{tabular}{lcccccc}
		\toprule
		& \multicolumn{3}{c}{deformazione} & \multicolumn{3}{c}{reazioni vincolari} \\
		\cmidrule(r){2-7}
		        & $\delta$ & $\theta$ & $\phi$ & $R_y$ & $M_x$ & $M_z$  \\
	simulazione	& $[mm]$  & $[mrad]$ & $[mrad]$ & $[N]$ & $[Nm]$ & $[Nm]$  \\
		\midrule
		Centine manovra & $-11.21$ & $12.8$ & $6.83$ & $2875$ & $-2315$ & $-367$  \\
		longh. manovra & $-10.82$ & $11.7$ & $-3.25$ & $2852$ & $-2264$ & $-292$  \\
		\midrule
		Centine raffica & $-8.15$ & $9.1$ & $3.20$ & $2696$ & $-1822$ & $-834$  \\
		longh. raffica & $-7.62$ & $8.2$ & $-0.18$ & $2640$ & $-1723$ & $-814$  \\				
		\bottomrule
	\end{tabular}
	\label{tab:res}
\end{table}


La sintesi dei risultati mostra come lo stabilizzatore sia sollecitato e si deformi flessionalmente. L'applicazione di carichi concentrati sulle centine garantisce anche una maggior rispondenza delle reazioni di incastro con i dati di progetto. \\
La stima dei carichi distribuiti sui longheroni risulta essere in difetto e dovrebbe essere rivista, anche se la differenza in termini di deformazione flessionale è minima. Inoltre, l'assenza di momento torcente nella parte esterna dello stabilizzatore porta a risultati contrastanti dal punto di vista dell'angolo di torsione.\\
Le figure \ref{fig:res1}, \ref{fig:res2} e \ref{fig:res3} mostrano i risultati ottenuti per il caso \emph{manovra}, mentre le figure \ref{fig:res4}, \ref{fig:res5} e \ref{fig:res6} mostrano i risultati ottenuti per il caso \emph{raffica}. \\
La figura \ref{fig:res7} mostra il livello di $\sigma$ di compressione nei pannelli inferiori.


\begin{figure}[htb]
	\centering
	\subfloat[carichi su centine]{\includegraphics[width=.99\columnwidth]{res/CM_displ.png}} \\
	\subfloat[carichi su longheorni]{\includegraphics[width=.99\columnwidth]{res/LM_displ.png}} \\
	\caption[manovra: spostamenti]{manovra: spostamenti} % The text in the square bracket is the caption for the list of figures while the text in the curly brackets is the figure caption
	\label{fig:res1}
\end{figure}

\begin{figure}[htb]
	\centering
	\subfloat[az. assiale nei correnti]{\includegraphics[width=.99\columnwidth]{res/CM_correnti_N.png}} \\
	\subfloat[$\sigma_{max}$ nei correnti]{\includegraphics[width=.99\columnwidth]{res/CM_correnti_sigma.png}} \\
	\caption[manovra: azioni negli elementi trave]{manovra: azioni negli elementi trave} % The text in the square bracket is the caption for the list of figures while the text in the curly brackets is the figure caption
	\label{fig:res2}
\end{figure}


\begin{figure}[htb]
	\centering
	\subfloat[VM stress $(\le 140Mpa)$ ]{\includegraphics[width=.99\columnwidth]{res/CM_VM01.png}} \\
	\subfloat[VM stress sup. $(\le 120Mpa)$]{\includegraphics[width=.49\columnwidth]{res/CM_VM02.png}} \
	\subfloat[VM stress inf. $(\le 60Mpa)$]{\includegraphics[width=.49\columnwidth]{res/CM_VM03.png}} \\
	\caption[manovra: VM stress]{manovra: VM stress} % The text in the square bracket is the caption for the list of figures while the text in the curly brackets is the figure caption
	\label{fig:res3}
\end{figure}

\begin{figure}[htb]
	\centering
	\subfloat[carichi su centine]{\includegraphics[width=.99\columnwidth]{res/CR_displ.png}} \\
	\subfloat[carichi su longheorni]{\includegraphics[width=.99\columnwidth]{res/LR_displ.png}} \\
	\caption[raffica: spostamenti]{raffica: spostamenti} % The text in the square bracket is the caption for the list of figures while the text in the curly brackets is the figure caption
	\label{fig:res4}
\end{figure}

\begin{figure}[htb]
	\centering
	\subfloat[az. assiale nei correnti]{\includegraphics[width=.99\columnwidth]{res/CR_correnti_N.png}} \\
	\subfloat[$\sigma_{max}$ nei correnti]{\includegraphics[width=.99\columnwidth]{res/CR_correnti_sigma.png}} \\
	\caption[raffica: azioni negli elementi trave]{raffica: azioni negli elementi trave} % The text in the square bracket is the caption for the list of figures while the text in the curly brackets is the figure caption
	\label{fig:res5}
\end{figure}


\begin{figure}[htb]
	\centering
	\subfloat[VM stress $(\le 120Mpa)$ ]{\includegraphics[width=.99\columnwidth]{res/CR_VM01.png}} \\
	\subfloat[VM stress sup. $(\le 120Mpa)$]{\includegraphics[width=.49\columnwidth]{res/CR_VM02.png}} \
	\subfloat[VM stress inf. $(\le 60Mpa)$]{\includegraphics[width=.49\columnwidth]{res/CR_VM03.png}} \\
	\caption[raffica: VM stress]{raffica: VM stress} % The text in the square bracket is the caption for the list of figures while the text in the curly brackets is the figure caption
	\label{fig:res6}
\end{figure}


\begin{figure}[htb]
	\centering
	\subfloat[manovra $\sigma_{xx}$ min]{\includegraphics[width=.99\columnwidth]{res/LM_sigma_xx.png}} \\
	\subfloat[raffica $\sigma_{xx}$ min]{\includegraphics[width=.99\columnwidth]{res/LR_sigma_xx.png}} \\
	\caption[pannelli a compressione]{pannelli a compressione} % The text in the square bracket is the caption for the list of figures while the text in the curly brackets is the figure caption
	\label{fig:res7}
\end{figure}

\newpage


%----------------------------------------------------------------------------------------
%	RESULTS AND DISCUSSION
%----------------------------------------------------------------------------------------

\section{Conclusioni}


\paragraph{I risultati} ottenuti mostrano come il comportamento della struttura sia prevalentemente flessionale e, come logico attendersi, il carico più critico è quello dovuto a manovra. La deformazione dello stabilizzatore risulta comunque contenuta, tale da non modificare l'operatività della superficie aerodinamica. \\
Le analisi evidenziano come la regione di attacco dello stabilizzatore sia quella maggiormente caricata. Per il modello adottato il livello degli sforzi risulta comunque contenuto al di sotto di limiti critici. \\
La validità e coerenza del modello è stata testata per i seguenti aspetti:

\begin{itemize}
	\item confronto con modello a semiguscio,
	\item verifica dei valori attesi di risultante e momento risultante complessivi delle forze applicate,
	\item verifica della corretta applicazione dei carichi tramite elementi di tipo \emph{RBE3} (anche attraverso modelli semplificati di supporto),
	\item verifica del corretto orientamento degli assi degli elementi di tipo \emph{beam},
	\item verifica del corretto orientamento degli assi principali degli elementi di tipo \emph{shell},
	\item verifica di qualità della mesh (presenza di nodi o \emph{edge} orfani, \emph{aspect ratio} degli elementi).
\end{itemize}

\paragraph{Analisi} ulteriori dovrebbero comprendere:

\begin{itemize}
	\item confronto con eventuali risultati sperimentali
	\item eventuale raffinamento della mesh, o utilizzo di elementi quadratici, per verificare l'assenza di picchi di sforzo che possono essere stati trascurati nell'analisi,
	\item studio di dettaglio della zona di incastro per verificare la resistenza della connessione alla fusoliera,
	\item valutazioni più approfondite sui pannelli soggetti a compressione per valutare eventuali fenomeni di instabilità,
	\item calcolo delle frequenze proprie e dei modi di vibrare principali.
\end{itemize}



%Reference to Table~\vref{tab:label}. % The \vref command specifies the location of the reference

%------------------------------------------------


%----------------------------------------------------------------------------------------
%	BIBLIOGRAPHY
%----------------------------------------------------------------------------------------

%\renewcommand{\refname}{\spacedlowsmallcaps{References}} % For modifying the bibliography heading

%\bibliographystyle{unsrt}

%\bibliography{sample.bib} % The file containing the bibliography

%----------------------------------------------------------------------------------------

\end{document}